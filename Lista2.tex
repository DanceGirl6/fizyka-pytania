\subsection{Na czym polega katastrofa ultrafioletowa?}

Eksperyment Stefana-Boltzmanna (1878) badał promieniowanie cieplne emitowane przez
ciało doskonale czarne. Ciało doskonale czarne to wyidealizowane ciało fizyczne pochłaniające całkowicie padające na nie promieniowanie elektromagnetyczne, niezależnie od temperatury tego ciała, kąta padania i widma padającego promieniowania. Według klasycznej teorii (prawo Rayleigha-Jeansa), ciało doskonale czarne powinno emitować coraz więcej energii w miarę wzrostu częstotliwości promieniowania. To prowadzi do absurdu – przy wysokich częstotliwościach (w zakresie nadfioletu i wyżej) energia promieniowania dąży do nieskończoności. Taki efekt nie występuje w rzeczywistości. Obserwacje eksperymentalne pokazały, że natężenie promieniowania nie rośnie w nieskończoność. Zgodnie z klasyczną fizyką, każda z nieskończonej liczby możliwych drgań wnęki ciała czarnego powinna otrzymywać pewną ilość energii. W rezultacie całkowita energia promieniowania miałaby być nieskończona – co jest fizycznie niemożliwe.

Katastrofa nadfioletowa była jednym z pierwszych dowodów, że klasyczna fizyka nie wystarcza do opisu zjawisk mikroświata. Jej rozwiązanie zapoczątkowało mechanikę kwantową.

\subsection{Jakie są eksperymentalne dowody tego, że światło istnieje, się emituje oraz jest absorbowane porcjami (kwantami)?}

Efekt fotoelektryczny (Hertz, Lenard, Einstein – 1887–1905) to jedno z najważniejszych i najwcześniejszych zjawisk potwierdzających, że światło oddziałuje z materią w sposób kwantowy. 

Efekt fotoelektryczny to zjawisko emisji elektronów z powierzchni metalu pod wpływem padającego na niego światła. Najważniejsze obserwacje:
\begin{itemize}
\item Elektrony są emitowane tylko wtedy, gdy częstotliwość światła przekracza pewną wartość progową, niezależnie od jego natężenia. 
\item Energia kinetyczna wybitych elektronów zależy od częstotliwości, a nie od intensywności światła.
\itemNie obserwuje się opóźnienia czasowego emisji nawet przy bardzo słabym świetle, jeśli tylko ma ono odpowiednią częstotliwość.
\end{itemize}

Einstein sformułował wzór efektu fotoelektrycznego
\begin{equation*}
\frac12 m v_{max}^2 = h \nu - W,
\end{equation*}
gdzie $W$ to funkcja pracy metalu (zależna od rodzaju metalu). Dowodzi to kwantyzacji: Elektron absorbuje energię światła w całości – nie stopniowo, lecz jednorazowo, jako pojedynczy kwant (foton).

\subsection{Przedyskutuj zjawisko interferencji fal}

Zjawisko interferencji fal polega na nakładaniu się dwóch lub więcej fal w przestrzeni, prowadzącym do wzmocnienia lub osłabienia amplitudy fali wypadkowej w danym punkcie.

W roku 1801 Thomas Young przeprowadził eksperyment, przepuszczając
przez szczeliny światło. Przez dwie szczeliny przechodziła tak zwana fala płaska, poruszająca się w kierunku ekranu. W sensie optyki klasycznej albo termodynamiki klasycznej, możemy powiedzieć, że przykładowo światło słoneczne jest taką falą płaską. Ta fala płaska przechodzi przez szczeliny, a następnie dalej jako fala płaska przemieszcza się w kierunku oddalonego ekranu. Co zobaczymy na ekranie? Na ekranie zobaczymy coś niespodziewanego - będzie to obraz interferencyjny.

\begin{itemize}
\item Amplituda fali -- wektor zależny od położenia w przestrzenii oraz od czasu: $$A(\overline{r}, t)$$
\item Intensywność światła
$$I = |A|^2$$
\item Zasada superpozycji -- aby obliczyć amplitudę całkowitą musimy zsumować amplitudy fal pochodzących z obu szczelin (z obu źródeł):
$$\overline{A}(\overline{r}, t) = \overline{A}_1(\overline{r}, t) + \overline{A}_2(\overline{r}, t)$$
\item Intensywność całkowita ma postać
$$I = |A_1|^2 + |A_2^2| + A_1 A_2^{*} + A_1^{*} A_2$$
Człon $A_1 A_2^{*} + A_1^{*} A_2$ jest odpowiedzialny za interferencję. Obraz widoczny na ekranie jest spowodowany superpozycją fal pochodzących z obu szczelin.
\end{itemize}

\subsection{Czym są pakiety falowe? Problem normalizacji, przekształcenia pomiędzy przestrzenią położenia oraz przestrzenią pędu.}

Pakiet falowy to reprezentacja cząstki jako superpozycji fal o różnych długościach fali. Wyraża się poprzez transformatę Fouriera jako suma fal płaskich.

W jednym wymiarze:
$$\Psi(x,t) = (2 \pi \hbar)^{-\frac12} \int_{-\infty}^{+\infty} \phi(p) \mathrm{e}^{\frac{i}{\hbar}\left(p x - E(p)t \right)} \mathrm{d}p,$$
gdzie wyrażenie pod całką to fala płaska, a $\phi(p)$ to funkcja określająca pakiet falowy.

W mechanice kwantowej funkcja falowa musi spełniać warunek normalizacji:
$$\int_{-\infty}^{+\infty} |\psi(x,t)|^2 \mathrm{d}x = 1,$$
co oznacza, że całkowite prawdopodobieństwo znalezienia cząstki gdziekolwiek = 1. 

Problem w tym, że fala płaska jest nienormalizowalna, bo jej wartość jest stała na całej osi x, nie da się jej sprowadzić do jednostkowego prawdopodobieństwa.

Rozwiązanie: Używamy pakietów falowych, czyli superpozycji fal płaskich o różnych $p$, które są lokalizowane w przestrzeni i dają się znormalizować.

Reprezentacja pędowa jest transformatą Fouriera funkcji falowej w położeniu:
$$\phi(p) = \frac{1}{\sqrt{2\pi \hbar}} \int_{-\infty}^{+\infty} \psi(x) \mathrm{e}^{-\frac{ipx}{\hbar}} \mathrm{d} x.$$
Funkcja falowa wyraża się więc przez odwrotną transformatę Fouriera reprezentacji pędowej:
$$\psi(x) = \frac{1}{\sqrt{2\pi \hbar}} \int_{-\infty}^{+\infty} \phi(p) \mathrm{e}^{\frac{ipx}{\hbar}} \mathrm{d} p.$$

\subsection{Stany kwantowe, operatory oraz równanie Schrödingera w interpretacji Feynmana}

odpowiedź

\subsection{Twierdzenie Ehrenfesta}

odpowiedź

\subsection{Obserwable, równanie Schrödingera, zależne oraz niezależne od czasu, własne stany, własne energie}

odpowiedź

\subsection{Proste zagadnienia: swobodna cząstka}

odpowiedź

\subsection{Proste zagadnienia: potencjał w kształcie schodków, bariery, studni kwadratowej}

odpowiedź

\subsection{Proste zagadnienia: oscylator harmoniczny}

odpowiedź

\subsection{Formalizm mechaniki kwantowej, postulaty}

odpowiedź

\subsection{Klasy i własności operatorów. Komutatory}

W mechanice kwantowej operatory reprezentują obserwowalne wielkości fizyczne, takie jak pozycja czy pęd. Operatory działają na przestrzeni stanów (np. w przestrzeni Hilberta) i mogą mieć różne własności: być hermitowskie (samosprzężone), jednostkowe czy projekcyjne. Hermitowskie operatory odpowiadają mierzalnym wartościom rzeczywistym.

Klasa operatorów określa ich charakter (np. ograniczone, nieograniczone). Ważnym pojęciem są komutatory operatorów
\[
[A, B] = AB - BA.
\]
Jeśli
\[
[A, B] = 0,
\]
to operatory się komutują, co oznacza, że można jednocześnie mierzyć odpowiadające im wielkości z pełną precyzją. Niezerowy komutator wskazuje na fundamentalne ograniczenia pomiarowe, jak w przypadku zasady nieoznaczoności Heisenberga.

\subsection{Zasada nieoznaczoności Heisenberga}

Zasada nieoznaczoności Heisenberga wyraża fundamentalne ograniczenie precyzji, z jaką można jednocześnie znać wartości pewnych par wielkości fizycznych, np. położenia \( \hat{x} \) i pędu \( \hat{p} \). Formalnie wyraża się to nierównością

\[
\Delta x \, \Delta p \geq \frac{\hbar}{2},
\]

gdzie \( \Delta x \) i \( \Delta p \) to odchylenia standardowe pomiarów operatorów położenia i pędu, a \( \hbar \) to zredukowana stała Plancka.

Ta zasada wynika z faktu, że operatory położenia i pędu nie komutują, tzn.

\[
[\hat{x}, \hat{p}] = i \hbar,
\]

co implikuje, że nie istnieje wspólny zbiór własnych wektorów obu operatorów, a więc nie można jednocześnie przypisać im dokładnych wartości.

\subsection{Operator momentu pędu, uogólniony operator momentu pędu, operator spinu: własne funkcje i wartości}


Operator momentu pędu \(\hat{\mathbf{L}} = (\hat{L}_x, \hat{L}_y, \hat{L}_z)\) opisuje moment pędu orbitalnego cząstki. Składowe operatora spełniają następujące relacje komutacyjne:

\[
[\hat{L}_i, \hat{L}_j] = i \hbar \epsilon_{ijk} \hat{L}_k,
\]

gdzie \(\epsilon_{ijk}\) to symbol Levi-Civity, a \(i, j, k \in \{x,y,z\}\).

Operator kwadrat momentu pędu \(\hat{L}^2 = \hat{L}_x^2 + \hat{L}_y^2 + \hat{L}_z^2\) oraz składowa \(\hat{L}_z\) mają wspólny układ własnych funkcji \(|l, m\rangle\), dla których zachodzą własności:

\[
\hat{L}^2 |l, m\rangle = \hbar^2 l(l+1) |l, m\rangle, \quad \hat{L}_z |l, m\rangle = \hbar m |l, m\rangle,
\]

gdzie \(l = 0, 1, 2, \ldots\), a \(m = -l, -l+1, \ldots, l\).

Uogólniony operator momentu pędu \(\hat{\mathbf{J}} = \hat{\mathbf{L}} + \hat{\mathbf{S}}\) łączy moment pędu orbitalny \(\hat{\mathbf{L}}\) oraz spin \(\hat{\mathbf{S}}\).

Operator spinu \(\hat{\mathbf{S}}\) opisuje wewnętrzny moment pędu cząstek, niezwiązany z ruchem orbitalnym. Składowe spinu również spełniają relacje komutacyjne analogiczne do momentu pędu:

\[
[\hat{S}_i, \hat{S}_j] = i \hbar \epsilon_{ijk} \hat{S}_k.
\]

Dla spinu \(s\) (np. \(s = \frac{1}{2}\) dla elektronu), własne wartości operatorów \(\hat{S}^2\) i \(\hat{S}_z\) są:

\[
\hat{S}^2 |s, m_s\rangle = \hbar^2 s(s+1) |s, m_s\rangle, \quad \hat{S}_z |s, m_s\rangle = \hbar m_s |s, m_s\rangle,
\]

gdzie \(m_s = -s, -s+1, \ldots, s\).

Własne funkcje momentu pędu i spinu tworzą bazę przestrzeni stanów kwantowych, na której można opisywać stan cząstki z uwzględnieniem zarówno ruchu orbitalnego, jak i spinu.

\subsection{Atom wodoru: stany własne oraz energie własne}


Atom wodoru opisuje równanie Schrödingera z potencjałem Coulomba:

\[
\hat{H} = -\frac{\hbar^2}{2m} \nabla^2 - \frac{e^2}{4 \pi \varepsilon_0 r},
\]

gdzie \(m\) to masa elektronu, \(e\) ładunek elementarny, a \(r\) odległość elektronu od jądra.

Stany własne \(|n, l, m\rangle\) są jednocześnie własnymi funkcjami operatorów:

\[
\hat{H}|n, l, m\rangle = E_n |n, l, m\rangle,
\]

\[
\hat{L}^2 |n, l, m\rangle = \hbar^2 l(l+1) |n, l, m\rangle,
\]

\[
\hat{L}_z |n, l, m\rangle = \hbar m |n, l, m\rangle,
\]

gdzie liczby kwantowe przyjmują wartości

\[
n = 1, 2, 3, \ldots, \quad l = 0, 1, \ldots, n-1, \quad m = -l, -l+1, \ldots, l.
\]

Energia własna jest określona wzorem:

\[
E_n = - \frac{m e^4}{2 (4 \pi \varepsilon_0)^2 \hbar^2} \frac{1}{n^2} = - \frac{13.6\,\mathrm{eV}}{n^2}.
\]

Energia zależy wyłącznie od głównej liczby kwantowej \(n\), co prowadzi do degeneracji poziomów energetycznych względem liczb \(l\) i \(m\).

\subsection{Własności funkcji falowych bozonów oraz fermionów}


W mechanice kwantowej funkcje falowe bozonów i fermionów różnią się symetrią względem zamiany dwóch identycznych cząstek:

\begin{itemize}
    \item \textbf{Bozony} mają symetryczne funkcje falowe, tzn. przy zamianie cząstek
    \[
    \Psi(\ldots, \mathbf{r}_i, \ldots, \mathbf{r}_j, \ldots) = + \Psi(\ldots, \mathbf{r}_j, \ldots, \mathbf{r}_i, \ldots).
    \]
    
    \item \textbf{Fermiony} mają antysymetryczne funkcje falowe, tzn.
    \[
    \Psi(\ldots, \mathbf{r}_i, \ldots, \mathbf{r}_j, \ldots) = - \Psi(\ldots, \mathbf{r}_j, \ldots, \mathbf{r}_i, \ldots).
    \]
\end{itemize}

Antysymetria funkcji falowej fermionów prowadzi do zasady Pauliego wykluczania, która zabrania zajmowania tego samego stanu kwantowego przez dwie identyczne fermiony.

Symetria funkcji falowej bozonów pozwala na zajmowanie tego samego stanu kwantowego przez wiele cząstek, co jest podstawą efektów takich jak kondensacja Bosego-Einsteina.
