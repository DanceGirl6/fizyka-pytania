\subsection{Jaka jest różnica między przestrzenią Hilberta a przestrzenią fizyczną?}

odpowiedź

\subsection{Czym jest operator liniowy w mechanice kwantowej?}

odpowiedź

\subsection{Zdefiniuj operator hermitowski.}

odpowiedź

\subsection{Jakie wielkości fizyczne odpowiadają operatorom hermitowskim?}

odpowiedź

\subsection{Czym są wartości własne i wektory własne w kontekście pomiarów kwantowych?}

odpowiedź

\subsection{Jakie jest fizyczne znaczenie iloczynu skalarnego w przestrzeni Hilberta?}

odpowiedź

\subsection{Wyjaśnij znaczenie normalizacji stanów kwantowych.}

odpowiedź

\subsection{Co oznacza, że dwa stany kwantowe są ortogonalne?}

odpowiedź

\subsection{Wyjaśnij pojęcie superpozycji stanów.}

odpowiedź

\subsection{Jakie są relacje komutacyjne operatorów położenia i pędu?}

odpowiedź

\subsection{Co oznacza, że dwa operatory komutują?}

odpowiedź

\subsection{Czym jest pełny układ bazowy w przestrzeni Hilberta?}

odpowiedź

\subsection{Co to jest operator unitarny?}

Operator liniowy w przestrzeni Hilberta, który zachowuje iloczyn skalarny.

\subsection{Jak transformacje unitarne zachowują prawdopodobieństwo?}

Transformacja unitarna $U$ spełnia $U^\dagger U = I$, więc zachowuje iloczyn skalarny:
\[
\langle \psi | \psi \rangle
\;\xrightarrow{\,U\,}\;
\langle \psi | U^\dagger U | \psi \rangle
= \langle \psi | \psi \rangle.
\]

Ponieważ norma wektora stanu (a więc suma wszystkich prawdopodobieństw) pozostaje $1$, każda amplituda przejścia 
$|\langle \phi | \psi \rangle|^{2}$ również się nie zmienia – stąd prawdopodobieństwa są zachowane.

\subsection{Wyjaśnij ewolucję czasową za pomocą operatora ewolucji.}

Ewolucja czasowa stanu układu kwantowego \(\lvert \psi(t) \rangle\) jest opisywana przez operator ewolucji \(U(t_{2},t_{1})\), który propaguje stan z czasu \(t_{1}\) do \(t_{2}\), zgodnie z zależnością
\[
\lvert \psi(t+\Delta t) \rangle \;=\; U\bigl(t+\Delta t,\,t\bigr)\,\lvert \psi(t) \rangle.
\]

\subsection{Wyjaśnij pojęcie stanów stacjonarnych.}

Stany stacjonarne to rozwiązania równania Schrödingera, w których funkcja falowa może być rozdzielona na część zależną od współrzędnych przestrzennych oraz część zależną od czasu,
\[
\Psi(\vec r, t) = \psi(\vec r)\,f(t)
\]
i opisują dozwolone poziomy energetyczne układu.

\subsection{Czym jest pakiet falowy?}

Pakiet falowy to zbiór fal, który można opisać funkcją falową \(\Psi(x,t)\) w postaci całki po różnych pędach:
\[
\Psi(x,t)\;=\;\int_{-\infty}^{\infty}\!\phi(p_x)\,e^{\tfrac{i}{\hbar}(p_x x - E t)}\,\mathrm{d}p_x,
\]
gdzie \(\phi(p_x)\) jest funkcją określającą kształt pakietu falowego.  
Przykładem jest pakiet Gaussowski, dla którego \(\phi(p_x)\) ma kształt funkcji Gaussa.

\subsection{Co oznacza transformacja Fouriera funkcji falowej i jakie ma znaczenie fizyczne?}

Transformacja Fouriera funkcji falowej pozwala na przejście z reprezentacji w przestrzeni położenia \(\Psi(x,0)\) do reprezentacji w przestrzeni pędu \(\phi(p_x)\) i odwrotnie, co umożliwia jednoczesne opisanie stanu cząstki zarówno w kategoriach położenia, jak i pędu:
\[
\phi(p_x)
=\frac{1}{\sqrt{2\pi\hbar}}
\int_{-\infty}^{\infty}
\Psi(x,0)\,e^{-\frac{i}{\hbar}p_x x}\,\mathrm{d}x,
\]
\[
\Psi(x,0)
=\frac{1}{\sqrt{2\pi\hbar}}
\int_{-\infty}^{\infty}
\phi(p_x)\,e^{\frac{i}{\hbar}p_x x}\,\mathrm{d}p_x.
\]

\subsection{Wyjaśnij zjawisko tunelowania kwantowego.}

\textbf{Źródło:} Notatki.pdf, rozdz.~5~\S5.3; Wikipedia\,\emph{Quantum tunnelling}.
		
		\vspace{4pt}
		\textbf{Opis zjawiska.}\  
		Jeżeli cząstka (np.~elektron) o energii $E$ zbliża się do skończonej bariery
		potencjału wysokości $V_0$ i szerokości $a$, klasyczna fizyka
		przewiduje całkowite odbicie, gdy $E<V_0$.
		W mechanice kwantowej funkcja falowa $\psi(x)$ przenika barierę i maleje
		wykładniczo:
		\[
		\psi(x)\propto e^{-\kappa x},\qquad \kappa=\sqrt{\tfrac{2m}{\hbar^{2}}(V_0-E)}.
		\]
		Współczynnik transmisji dla bariery jednorodnej ($\kappa a\gg1$):
		\[
		T\simeq e^{-2\kappa a},
		\]
		co wyjaśnia działanie STM, $\alpha$-rozpadu i prądów Josephsona.
		
		\vspace{4pt}
		\textbf{Co oznacza co?}
		\begin{itemize}
			\item $m$ – masa cząstki,
			\item $\hbar$ – zredukowana stała Plancka,
			\item $V_0$, $a$ – wysokość i szerokość bariery,
			\item $\kappa$ – współczynnik tłumienia w barierze,
			\item $T$ – prawdopodobieństwo przejścia (transmisji), z $R+T=1$.
		\end{itemize}

\subsection{Wyjaśnij transformację Fouriera między przestrzenią położeń a pędu.}

odpowiedź

\subsection{Czym są stany swobodne i związane?}

odpowiedź

\subsection{Jakie są warunki ciągłości dla funkcji falowej i jej pochodnych?}

odpowiedź

\subsection{Co to jest operator momentu pędu?}

odpowiedź

\subsection{Jakie są relacje komutacyjne dla składników momentu pędu?}

odpowiedź

\subsection{Jakie są wartości własne $L^2$ i $L_z$?}

\begin{align*}
L^2 Y_{lm} &= \hbar^2 l(l+1) Y_{lm}, \\
L_z Y_{lm} &= \hbar m Y_{lm},
\end{align*}
gdzie:
\begin{itemize}
  \item \( Y_{lm}(\theta, \phi) \) -- sferyczna funkcja harmoniczna,
  \item \( \hbar \) -- zredukowana stała Plancka, \( \hbar = \frac{h}{2\pi} \),
  \item \( l \in \{0, 1, 2, \ldots\} \) -- główna (orbitalna) liczba kwantowa,
  \item \( m \in \{-l, -l+1, \ldots, l\} \) -- magnetyczna liczba kwantowa.
\end{itemize}

\subsection{Jaki jest związek między funkcjami $Y_{lm}$ a momentem pędu?}

Funkcje \( Y_{lm}(\theta, \phi) \) to sferyczne funkcje harmoniczne. Są funkcjami własnymi operatorów momentu pędu. Funkcje \( Y_{lm} \) opisują część kątową funkcji falowej w układach sferycznie symetrycznych, np. w atomie wodoru.

\subsection{Co to jest obrót w przestrzeni i jak działa na funkcje falowe?}

Obrót w przestrzeni to transformacja układu współrzędnych (lub funkcji) polegająca na obróceniu obiektów względem punktu (zwykle początku układu współrzędnych) o pewien kąt wokół wybranej osi. Stan układu opisuje funkcja falowa $\psi(\vec{r})$. Obrót przestrzeni wpływa na sposób, w jaki ta funkcja jest rozłożona w przestrzeni.

\subsection{Co to jest moment własny (spin)?}

Moment własny, czyli \textbf{spin}, to wewnętrzna, kwantowa forma momentu pędu cząstki. Spin jest cechą wrodzoną cząstki i nie wynika z jej ruchu w przestrzeni. Opisywany jest operatorem $\vec{S}$. Na przykład cząstka w eksperymencie Sterna-Gerlacha ma spin \( s = 1 \), więc może przyjmować trzy stany spinu: \( m_s = -1, 0, +1 \).

\subsection{Jakie są wartości dozwolone dla spinu cząstki?}

Spin cząstki, może przyjmować wartości:
\[
s \in \left\{ 0, \tfrac{1}{2}, 1, \tfrac{3}{2}, 2, \dots \right\}.
\]

\subsection{Czym są macierze Pauliego?}

Macierze Pauliego to zestaw trzech specjalnych macierzy \( 2 \times 2 \), oznaczanych zwyczajowo jako \( \sigma_x \), \( \sigma_y \) oraz \( \sigma_z \). Stanowią one podstawę do opisu operatorów spinu dla cząstek o spinie \( \frac{1}{2} \). Mają postać:

\[
\sigma_x = \begin{pmatrix}
0 & 1 \\
1 & 0
\end{pmatrix}, \quad
\sigma_y = \begin{pmatrix}
0 & -i \\
i & 0
\end{pmatrix}, \quad
\sigma_z = \begin{pmatrix}
1 & 0 \\
0 & -1
\end{pmatrix}.
\]

Pozwalają na opis stanów spinu i ich transformacji, m.in. w eksperymentach Stern-Gerlacha czy przy analizie obrotów kwantowych.

\subsection{Zdefiniuj operator spinowy $S$.}

odpowiedź

\subsection{Co to jest doświadczenie Stern-Gerlacha i co pokazuje?}

odpowiedź

\subsection{Co to znaczy, że spin nie ma klasycznego odpowiednika?}

odpowiedź

\subsection{Jakie są stany własne atomu wodoru?}

odpowiedź

\subsection{Jakie są liczby kwantowe w atomie wodoru?}

odpowiedź

\subsection{Wyjaśnij degenerację poziomów energetycznych w atomie wodoru.}

odpowiedź

\subsection{Na czym polega zakaz Pauliego?}

odpowiedź

\subsection{Czym jest symetria permutacji fermionów i bozonów?}

odpowiedź

\subsection{Czym jest model Bohra?}

odpowiedź

\subsection{Czym jest energia jonizacji?}

odpowiedź

\subsection{Czym jest efekt fotoelektryczny?}

odpowiedź
